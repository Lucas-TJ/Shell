%%%%%%%%%%%%%%%%%%%%%%% file typeinst.tex %%%%%%%%%%%%%%%%%%%%%%%%%
%
% This is the LaTeX source for the instructions to authors using
% the LaTeX document class 'llncs.cls' for contributions to
% the Lecture Notes in Computer Sciences series.
% http://www.springer.com/lncs       Springer Heidelberg 2006/05/04
%
% It may be used as a template for your own input - copy it
% to a new file with a new name and use it as the basis
% for your article.
%
% NB: the document class 'llncs' has its own and detailed documentation, see
% ftp://ftp.springer.de/data/pubftp/pub/tex/latex/llncs/latex2e/llncsdoc.pdf
%
%%%%%%%%%%%%%%%%%%%%%%%%%%%%%%%%%%%%%%%%%%%%%%%%%%%%%%%%%%%%%%%%%%%


\documentclass[runningheads,a4paper]{llncs}

\usepackage[colorlinks=true,linkcolor=black,citecolor=black,pdfnewwindow=true,urlcolor=blue,filecolor=blue,baseurl=./]{hyperref}
\usepackage{url}

% \newcommand{\mycite}[2]{\url{#1}{\cite{#2}}}
% \newcommand{\mycite}[2]{\url{#1}:\cite{#2}}
% \newcommand{\mycite}[2]{\href{http://twitter.com/home}{Twitter}.}
%\newcommand{\mycite}[2]{\href{./refs/#1}{\cite{#2}}}
\newcommand{\mycite}[2]{\cite{#2}}
%\newcommand{\myciteurl}[2]{\href{#1}{\cite{#2}}}


\usepackage{amssymb}
\setcounter{tocdepth}{3}
\usepackage{graphicx}

\usepackage{url}
%\urldef{\mailsa}\path|{alfred.hofmann, ursula.barth, ingrid.haas, frank.holzwarth,|
%\urldef{\mailsb}\path|anna.kramer, leonie.kunz, christine.reiss, nicole.sator,|
%\urldef{\mailsc}\path|erika.siebert-cole, peter.strasser, lncs}@springer.com|    
%\newcommand{\keywords}[1]{\par\addvspace\baselineskip
%\noindent\keywordname\enspace\ignorespaces#1}

\usepackage{color}
\usepackage{amsmath}
\usepackage[tight]{subfigure}
\usepackage{listings}
\definecolor{darkgray}{rgb}{0.2,0.2,0.2}

\lstset{language=c++,basicstyle=\scriptsize\sffamily,tabsize=2,framexleftmargin=5mm, frame=shadowbox, framesep=3pt,rulesepcolor=\color{darkgray},rulesep=.100pt,keywordstyle=\bf\color{blue},commentstyle=\color{magenta},stringstyle=\color{red},numbers=left,numberstyle=\tiny,numbersep=10pt,breaklines=true,showstringspaces=false,columns=flexible}

\newcommand{\tomas}[1]{{\color{red}{\textbf{Tom: #1}}}}
\newcommand{\stefan}[1]{{\color{magenta}{\textbf{St: #1}}}}
\newcommand{\christian}[1]{{\color{green}{\textbf{Ch: #1}}}}
\newcommand{\tobias}[1]{{\color{bleu}{\textbf{Tob: #1}}}}


\newcommand{\myvec}[1]{\ensuremath{\mathbf{#1}}}
\newcommand{\mymat}[1]{\ensuremath{\mathbf{#1}}}
\newcommand{\Vx}{\myvec{x}} % position vector
\newcommand{\Vv}{\myvec{v}} % velocity vector
\newcommand{\Va}{\myvec{a}} % acceleration vector
\newcommand{\Vdv}{\myvec{dv}} % change of velocity vector (unknown in implicit CG, and used in constraint solver)
\newcommand{\deriv}[2]{\frac{\partial #1}{\partial #2}} % Partial derivative


% \newcommand{\mysubsection}[1]{\noindent \textbf{#1: }}
\newcommand{\mysubsection}[1]{\textbf{#1: }}
% \newcommand{\mysubsection}[1]{\subsection{#1}}

%\usepackage{miseenpage}

\begin{document}

\mainmatter            % start of your contributions

\title{Towards planning of infantile cardiac surgery based on simulation}

\author{PaperID:} % TODO

\maketitle

\begin{abstract}
% TODO
\end{abstract}

\section{Introduction}
State of the Art

Figure illustrating a surgery

Requirements: Model blood vessels/wall deformation, surgical low-level tasks

Plan

\section{Blood Vessel Model} % {{{

Thin shell model
\begin{itemize}
  \item based on mechanics of continuum and shell theory
  \item used in physics for deformation modelling
  \item briefly, refer to some textbook
\end{itemize}

\noindent Eq. showing dependence on rest position:
simple $\myvec{f} = K(\myvec{x} - \myvec{x_0})$

\noindent Fulfills requirements + collision
\begin{itemize}
  \item we need deformations: like bending and twisting
  \item deformation should be global
  \item should allow arbitrary models and not rely e.g. on tubular
  structure (like \cite{Li2009})
  \item the model has to be able to deal with collisions (e.g. between AA
  and PA)
\end{itemize}

\noindent Test with tube + convergence
\begin{itemize}
  \item experimented with Olivier's model \cite{Comas2010b,Comas2010c} and
  our own model, in principle the shell implementation is not that
  important
  \item geometrical tool, independence on physical parameters -- E
\end{itemize}

% }}}

\section{Simulation of Surgical Procedures}

\noindent surgical low-level tasks done with rest position trick (joining, patch, complex cutting/suturing)
\begin{itemize}
  \item describe the particular tasks and show that the essence is the same,
  \item describe our method with all the tricks
  \item include a schematic illustration
\end{itemize}

\section{Results and Discussion}

Example with 3 diff. surgical procedures

Parameters

Limitations (getting image data, meshes, dynamic remeshing, same number of nodes)

\section{Conclusion}

%\begin{abstract}
%% TODO
%In this paper...
%\end{abstract}
%
%\section{Introduction} % {{{
%
%%%%%%%
%%
%%\christian{We can put any comment...}
%%\tomas{People have different colors}
%%
%%Position vector: $\Vx$
%%
%%Matrices are written as $\mymat{M}$
%%
%%%%%%%
%
%The congenital heart diseases have become quite common \tomas{\ldots}.
%While they are very much treatable there are a lot of difficulties involved
%with the process. There is rarely only a single flaw, the lesions usually
%appear in multiples. Since most of the lesions would be fatal if untreated
%the first surgery has to be performed shortly after the birth, often
%followed with several other surgeries before the age of one year. At such
%young age the heart and blood vessels are very small which requires skills
%of the surgeon. Further more many of the critical decisions cannot be
%concluded from the CT scan or artificial model and have to be done on site
%(e.g. how exactly perform the cut and stitching). This is however in direct
%contrast with the requirement on the speed at which the operation has to be
%performed. During the surgery the blood flow has to be sustained by other
%means (\tomas{pig's heart?}) which puts overall pressure on the organism
%and thus cannot be kept for too long.
%
%\tomas{interesting ref. \cite{Barron2012}}
%
%\tomas{Stefan, please review this. It is just the notion that I have got
%about the problem and might not be completely correct}
%
%For these reasons there is a demand for tools helping with the
%process of pre-operative planning.
%\tomas{Usefulness of computer aid --
%numerical models:
%3D visualisation: Hemminger2005}
%
%\tomas{Are there other approaches to this or similar problem? My search
%did not reveal anything interesting.}
%
%Some of the most common actions in the procedure are
%
%\begin{itemize}
%    \item cutting out a piece of a blood vessel to remove an obstruction or
%    unwanted junction,
%    \item opening a narrow vessel by a cut and closing the hole by an
%    artificial material.
%    \item \tomas{more ideas?}
%\end{itemize}
%
%\tomas{add images}
%
%Both procedures are complicated from the surgeons point of view. In the
%first case the surgeon has to make an educated guess how to perform a cut
%so that after stitching the rest together the blood flow is not obstructed
%and the blood pressure at the seam is not too large so as to tear open the
%vessel. In the second case the surgeon has no prior idea about the shape
%and size of the patch he will need until he performs the cut. Therefor the
%patch has to be cut on site and cannot be prepared in advance.
%
%We are proposing new approach to pre-operative planning of cardiac surgery.
%Using shell elements to model deformations of the blood vessel we help with
%the decision making.
%
%% }}}
%
%\section{Elastic Shell Elements}
%
%To model the blood vessels we employed shell elements in the corotational
%environment. The main advantage of shells is their physical accuracy. They
%are based on the mechanics of continuum. Also the parameters defining the
%parameters of the simulation have clear physical interpretation (unlike for
%example for mass-spring model).
%
%
%The common objection against using shell elements in real-time simulations
%is the numerical complexity. While this is technically valid, it is not
%good enough reason. Using low resolution mesh we can obtain results with
%good visual and physical accuracy sufficient for real-time
%simulation.\tomas{isn't this too strong?} Given the bending nature of the
%shell one can employ interpolation of the shell surface to improve the
%visual mesh presented to the user. \tomas{cite Olivier and our unpublished
%results}
%
%A shell element is a combination of two other elements: elastic membrane
%and bending plate element. While the membrane element is used to compute
%in-plane forces of the shell (like stretching or shearing), the bending
%plate elements on the other hand describes the bending properties.
%
%\tomas{add image showing the difference}
%
%\tomas{thin in one dimension}
%
%\subsection{Mechanics of Continuum Preliminaries} % {{{
%
%\tomas{recheck this. btw do we really want this?}
%
%The mathematical formulation follows the steps defined by the mechanics of
%continuum.\tomas{ref. to some text book}
%
%For every particle $\myvec{x} \in \Omega$ of the deformable body $\Omega$
%the deformation $ \phi : \Omega' \to \Omega $ is a mapping from the rest
%shape $\myvec{x}' \in \Omega'$ to it's current shape
%
%$$ \phi(x) = \myvec{x}' + u(x) $$
%
%Where $u(x)$ is the displacement for particle $x$. From the Cauchy's strain
%tensor
%
%$$
%\epsilon = \frac{1}{2}\left( \nabla^T u + \nabla u \right)
%$$
%
%\noindent%
%we can compute the stress by applying a constitutive law. In our simulation
%we are using linear Hooke's law to keep the numerical formulation simple
%enough to be able to compute it in real-time. The stress is computed by
%
%$$
%\sigma = \mymat{M}\epsilon
%$$
%
%\noindent%
%where $\mymat{M}$ is the material matrix. Further from the stress and
%strain we can compute the strain energy
%
%$$
%E = \int_{\Omega} \epsilon^T \sigma \,\mathrm{d}\Omega
%$$
%
%\tomas{now should be discretization}
%
%%}}}
%
%\subsection{shells...} % {{{
%
%In this subsection we will briefly summarize how the numerical formulation
%of the linear model is derived. For more thorough description we refer the
%reader to \tomas{add ref.} Let's define $ u_x, u_y $ to be the two
%in-plane displacements and $ u_z $ the off-plane function of deflection. We
%define the strain-displacement matrices in the following way. For the
%membrane element (using the Cauchy's strain tensor)
%
%\begin{equation}
%  \mymat{B_m} = \left[ \begin{matrix}
%    \deriv{u_x}{x} \\
%    \deriv{u_y}{y} \\
%    \deriv{u_x}{y} + \deriv{u_y}{x}
%  \end{matrix} \right]
%\end{equation}
%
%\noindent%
%and for the bending plate element (using the Kirchoff-Love theory)
%
%\begin{equation}
%  \mymat{B_b} = \left[ \begin{matrix}
%    - z \deriv{^2 u_z}{x^2} \\
%    - z \deriv{^2 u_z}{y^2} \\
%    - 2z \deriv{^2 u_z}{xy}
%  \end{matrix} \right]
%\end{equation}
%
%From these we can easily compute the stiffness matrices $K_m$ and $K_b$
%(for membrane and bending plate element respectively) by integrating over
%the volume of the shell
%
%\begin{equation}
%  \mymat{K}_i = \int_V \mymat{B}_i^T \mymat{M} \mymat{B}_i \,\mathrm{d}V
%\end{equation}
%
%\noindent%
%where $i \in \{\mathrm{b},\mathrm{m}\}$ and $\mymat{M}$ is the material
%matrix defined by Hooke's law:
%
%\begin{equation}
%  \mymat{M} = \frac{E}{1-\nu^2} \left[ \begin{matrix}
%    1   & \nu & 0 \\
%    \nu & 1   & 0 \\
%    0   & 0   & \frac{1-\nu}{2}
%  \end{matrix} \right]
%\end{equation}
%
%The only two values defining the matrix value are Poisson's ratio $\nu$ and
%the Young's modulus $E$.
%
%% }}}
%
%\subsection{Co-rotational Formulation}
%
%While the Cauchy's strain used in the formulation of the shell element is
%invariant to rigid body translation, it is known to be rotationally
%noninvariant. This produces ghost deformations just by rigid rotation. One
%way to solve the problem is using Green's strain tensor instead whose
%definition is, however, non-linear. Therefore we to overcome the problem we
%chose to use the corotational formulation instead.\tomas{add ref;
%maybe short description?}
%
%
%\section{Method Description} % {{{
%
%\tomas{clearly state that this is not a complete solution where user points
%and clicks -- it's just a proof-of-concept}
%
%In this chapter we will elaborate on the internal aspects of the
%simulation.
%
%Outline:
%
%\begin{itemize}
%    \item we use shell elements to model blood vessels, dynamic system
%        (implicit integration), in Sofa framework \tomas{cite Olivier
%        \cite{Comas2010b,Comas2010c} -- shells in corotational environment}
%    \item using springs to connect the objects is not good --> generates
%        excessive amount of energy
%    \item to eliminate this we use one mesh for the simulation and
%        different mesh to describe the rest shape and we leave the dynamic
%        system to stabilize
%        -- we need to map the nodes between topologies
%    \item to reduce more the generation of excessive energies in the system
%        at the beginning of the simulation we repetitively change the rest
%        shape (linear interpolation)
%\end{itemize}
%
%We are aiming for non-interactive simulation without requiring the
%surgeon to go through the whole process of the surgery. This introduces several
%problems we have to deal with.
%
%The first problem is how to deform the model so that the requested parts
%can be sutured together. And second how to simulate the suture.
%The simplest solution here would be connecting the respective nodes
%together with springs. Such solution, however, has a disadvantage of adding
%excessive energies to the system.\tomas{do we need a ref.?}
%
%We propose a different approach which uses a two different mesh topologies.
%The basic idea is to transform the mesh by joining the edges and nodes
%where the suturing is supposed to take place and let the dynamic system
%find a stable configuration. Such stable configuration is the result we are
%looking for. This requires us to provide a mapping to the underlying shell
%element code that would translate between nodes of the topology of the rest
%shape and the topology of the deformed mesh. Such a mapping is 1:N for
%nodes where the suturing takes place. \tomas{add image} To make it properly
%defined elements have to be taken into account also:
%
%\begin{equation}
%  \phi : N_d \times E_d -> N_r
%\end{equation}
%
%In the definition of this mapping $\phi$ the $N_d$ and $E_d$ are sets of
%nodes and elements of the deformed mesh and $N_r$ is set of nodes of the
%deformed mesh.
%
%\tomas{what more?}
%
%
%Another problem that a predictive simulation should deal with is filling
%the missing steps that would normally had to be performed fluently by the
%surgeon. For example when suturing two blood vessels together they have to
%be placed close to one another first. Sometimes relatively large distances
%have to be overcome. By ignoring this fact our approach with joining nodes
%and edges could generate unwanted forces at the start of the simulation. To
%alleviate large initial deformations for the elements we employed
%interpolation ...
%
%
%% }}}
%
%\section{Conclusion}
%%\subsection{Verification}
%
%Verification is difficult -- limited or no knowledge of real state after
%the surgery (too risky to put the child under CT/MRI again). 
%
%Therefore we are interested in the geometrical evaluation of the result
%i.e. no excessive elongations, expected deformations, independence on
%physical parameters (E/nu),\ldots or something like that \ldots


\bibliographystyle{abbrv}
\bibliography{miccai2012}

\end{document}

% Some Vim setup. If it interferes with anyones work feel free to remove
% that.
% vim: et sw=2 tw=75 fdm=marker fdc=2 spell
