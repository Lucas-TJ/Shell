%%%%%%%%%%%%%%%%%%%%%%% file typeinst.tex %%%%%%%%%%%%%%%%%%%%%%%%%
%
% This is the LaTeX source for the instructions to authors using
% the LaTeX document class 'llncs.cls' for contributions to
% the Lecture Notes in Computer Sciences series.
% http://www.springer.com/lncs       Springer Heidelberg 2006/05/04
%
% It may be used as a template for your own input - copy it
% to a new file with a new name and use it as the basis
% for your article.
%
% NB: the document class 'llncs' has its own and detailed documentation, see
% ftp://ftp.springer.de/data/pubftp/pub/tex/latex/llncs/latex2e/llncsdoc.pdf
%
%%%%%%%%%%%%%%%%%%%%%%%%%%%%%%%%%%%%%%%%%%%%%%%%%%%%%%%%%%%%%%%%%%%


\documentclass[runningheads,a4paper]{llncs}

\usepackage[colorlinks=true,linkcolor=black,citecolor=black,pdfnewwindow=true,urlcolor=blue,filecolor=blue,baseurl=./]{hyperref}
\usepackage{url}

% \newcommand{\mycite}[2]{\url{#1}{\cite{#2}}}
% \newcommand{\mycite}[2]{\url{#1}:\cite{#2}}
% \newcommand{\mycite}[2]{\href{http://twitter.com/home}{Twitter}.}
%\newcommand{\mycite}[2]{\href{./refs/#1}{\cite{#2}}}
\newcommand{\mycite}[2]{\cite{#2}}
%\newcommand{\myciteurl}[2]{\href{#1}{\cite{#2}}}


\usepackage{amssymb}
\setcounter{tocdepth}{3}
\usepackage{graphicx}

\usepackage{url}
%\urldef{\mailsa}\path|{alfred.hofmann, ursula.barth, ingrid.haas, frank.holzwarth,|
%\urldef{\mailsb}\path|anna.kramer, leonie.kunz, christine.reiss, nicole.sator,|
%\urldef{\mailsc}\path|erika.siebert-cole, peter.strasser, lncs}@springer.com|    
%\newcommand{\keywords}[1]{\par\addvspace\baselineskip
%\noindent\keywordname\enspace\ignorespaces#1}

\usepackage{color}
\usepackage{amsmath}
\usepackage[tight]{subfigure}
\usepackage{listings}
\definecolor{darkgray}{rgb}{0.2,0.2,0.2}

\lstset{language=c++,basicstyle=\scriptsize\sffamily,tabsize=2,framexleftmargin=5mm, frame=shadowbox, framesep=3pt,rulesepcolor=\color{darkgray},rulesep=.100pt,keywordstyle=\bf\color{blue},commentstyle=\color{magenta},stringstyle=\color{red},numbers=left,numberstyle=\tiny,numbersep=10pt,breaklines=true,showstringspaces=false,columns=flexible}

\newcommand{\tomas}[1]{{\color{red}{\textbf{Tom: #1}}}}
\newcommand{\stefan}[1]{{\color{magenta}{\textbf{St: #1}}}}
\newcommand{\christian}[1]{{\color{green}{\textbf{Ch: #1}}}}
\newcommand{\tobias}[1]{{\color{bleu}{\textbf{Tob: #1}}}}


\newcommand{\myvec}[1]{\ensuremath{\mathbf{#1}}}
\newcommand{\mymat}[1]{\ensuremath{\mathbf{#1}}}
\newcommand{\Vx}{\myvec{x}} % position vector
\newcommand{\Vv}{\myvec{v}} % velocity vector
\newcommand{\Va}{\myvec{a}} % acceleration vector
\newcommand{\Vdv}{\myvec{dv}} % change of velocity vector (unknown in implicit CG, and used in constraint solver)





% \newcommand{\mysubsection}[1]{\noindent \textbf{#1: }}
\newcommand{\mysubsection}[1]{\textbf{#1: }}
% \newcommand{\mysubsection}[1]{\subsection{#1}}

%\usepackage{miseenpage}

\begin{document}

\mainmatter            % start of your contributions

\title{Towards planning of infantile cardiac surgery based on simulation}

\author{PaperID:} % TODO

\maketitle

\begin{abstract}
% TODO
In this paper...
\end{abstract}

\section{Introduction} % {{{

%%%%%%
%
%\christian{We can put any comment...}
%\tomas{People have different colors}
%
%Position vector: $\Vx$
%
%Matrices are written as $\mymat{M}$
%
%%%%%%

The congenital heart diseases have become quite common \tomas{\ldots}.
While they are very much treatable there are a lot of difficulties involved
with the process. There is rarely only a single flaw, the lesions usually
appear in multiples. Since most of the lesions would be fatal if untreated
the first surgery has to be performed shortly after the birth, often
followed with several other surgeries before the age of one year. At such
young age the heart and blood vessels are very small which requires skills
of the surgeon. Further more many of the critical decisions cannot be
concluded from the CT scan or artificial model and have to be done on site
(e.g. how exactly perform the cut and stitching). This is however in direct
contrast with the requirement on the speed at which the operation has to be
performed. During the surgery the blood flow has to be sustained by other
means (\tomas{pig's heart?}) which puts overall pressure on the organism
and thus cannot be kept for too long.

\tomas{interesting ref. \cite{Barron2012}}

\tomas{Stefan, please review this. It is just the notion that I have got
about the problem and might not be completely correct}

For these reasons there is a demand for tools helping with the
process of pre-operative planning.

\tomas{Are there other approaches to this or similar problem? My search
did not reveal anything interesting.}

Some of the most common actions in the procedure are

\begin{itemize}
    \item cutting out a piece of a blood vessel to remove an obstruction or
    unwanted junction,
    \item opening a narrow vessel by a cut and closing the hole by an
    artificial material.
    \item \tomas{more ideas?}
\end{itemize}

\tomas{add images}

Both procedures are complicated from the surgeons point of view. In the
first case the surgeon has to make an educated guess how to perform a cut
so that after stitching the rest together the blood flow is not obstructed
and the blood pressure at the seam is not too large so as to tear open the
vessel. In the second case the surgeon has no prior idea about the shape
and size of the patch he will need until he performs the cut. Therefor the
patch has to be cut on site and cannot be prepared in advance.

We are proposing new approach to pre-operative planning of cardiac surgery.
Using shell elements to model deformations of the blood vessel we help with
the decision making.

% }}}

\tomas{What to describe? -- shell elements/FEM/mechanics?}
\tomas{shortly mention: rest shape}

\section{Method Description}                          

Outline:

\begin{itemize}
    
    \item we use shell elements to model blood vessels, dynamic system
        (implicit integration), in Sofa framework \tomas{cite Olivier
        \cite{Comas2010b,Comas2010c} -- shells in corotational environment}
    \item using springs to connect the objects is not good --> generates
        excessive amount of energy
    \item to eliminate this we use one mesh for the simulation and
        different mesh to describe the rest shape and we leave the dynamic
        system to stabilize
    \item to reduce more the generation of excessive energies in the system
        at the beginning of the simulation we repetitively change the rest
        shape (linear interpolation)
\end{itemize}


\section{Conclusion}
\subsection{Verification}

Verification is difficult -- limited or no knowledge of real state after
the surgery (can't put the child under CT/MRI again). \ldots


\bibliographystyle{abbrv}
\bibliography{miccai2012}

\end{document}

% Some Vim setup. If it interferes with anyones work feel free to remove
% that.
% vim: et tw=75 fdm=marker spell
