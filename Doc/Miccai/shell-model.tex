Several different models are commonly used for modelling thin structures or
tubular-like structures resembling blood vessels. \tomas{Add refs. or
reformulate!}
We choose to use thin shell elements for the predictive
simulation of the surgery. Shell elements are often considered numerically
too expensive, this is however not necessarily true.

The formulation of the thin shell elements is based on the mechanics of
continuum and these elements are commonly used in physics and mechanics for
deformation modelling of thin structures.

\tomas{TODO}
\noindent
briefly, refer to some textbook,
Eq. showing dependence on rest position:
simple $\myvec{f} = K(\myvec{x} - \myvec{x_0})$

\ldots

From our point of view, there are several requirements that the model
should be able to fulfill. Namely:

\begin{itemize}
  \item good support for deformations by bending and twisting
  \item the results of the deformation should be global
  \item method should allow arbitrary models and not rely on specific
  structure of the object (e.g. on tubular structure like \cite{Li2009})
  \item the model has to be able to deal with collisions (e.g. between the
  aortic arch and pulmonary artery)
\end{itemize}

Simulation using shell model satisfies all these requirements very well.

\tomas{TODO}
\noindent Test with tube + convergence
\begin{itemize}
  \item experimented with Olivier's model \cite{Comas2010b,Comas2010c} and
  our own model, in principle the shell implementation is not that
  important
  \item geometrical tool, independence on physical parameters -- E
\end{itemize}
