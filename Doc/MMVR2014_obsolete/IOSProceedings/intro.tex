\section*{Introduction}%
According to the statistics, nearly 100,000 European citizens die of cirrhosis or liver cancer each year. 
Although new methods 
%such as radio-frequency- and cryo-ablation known 
in the interventional radiology 
seem to be promising, surgery remains the option that offers the foremost success rate against these pathologies. 
Nevertheless, surgery is not always performed due to several limitations, in particular the determination 
of accurate eligibility criteria for the patient. 
%In this context, the pre-operational planning becomes a crucial task having a significant impact on the treatment. 

Computer-aided physics-based medical simulation has proven to be an extremely useful technique in the area of medical training. 
Whereas generic models are usually required in training simulators, accurate patient-specific modeling
becomes necessary as soon as computer simulation is to be employed in the pre-operative planning. 
%At the same time, 
%interactivity of such models remains an important aspect, requiring real-time simulation which is often difficult to 
%achieve given the complexity of soft tissues. 
When simulating the behavior of human liver, the task of real-time accurate modeling is challenging, mainly because of the complex structure 
of this organ composed of three main constituents: \emph{parenchyma}, \emph{vascular networks} and \emph{Glisson's capsule}.
The parenchyma has certainly been the most studied component of the liver; actually researchers agree on hyperelastic 
properties of the tissue, for which the mechanical parameters have been reported for example in~\cite{Kerdok2006}.%,Gao2009}. 
Moreover, several methods have been proposed to model the hyperelastic behaviour at real-time rates, such as multiplicative Jacobian decomposition
introduced in~\cite{Marchesseau2010}.
%
%\TG{The following paragraph is optional. Unnecessary but adds to the global context}
The mechanical importance of the vascular structures in liver is studied in~\cite{Peterlik2012}. It shows that the 
influence of vessels on the mechanical behaviour of the organ is significant. %, mainly if large deformations occur. 
%Since a detailed modeling of the vessels would be extremely costly (mainly because of the small thickness of the vessel wall, 
%the authors propose a composite model allowing for real-time simulation of entire liver with vascularization.
Since a detailed modeling of the vessels would be extremely costly (mainly because of the small thickness of the vessel wall), 
the authors propose a composite model allowing for real-time simulation.

Rather a small number of studies have been conducted dealing with the third liver constituent, the  Glisson's capsule.
Quantitative results of experiments on a porcine liver have been published in~\cite{Umale2011}; the measurements indicate that although being very 
thin (10--20$\mu m$), the capsule shows to be stiff in tensile tests: the Young's modulus of the capsule reported to be $8.22\pm3.42$\,MPa 
exceeds the values for the parenchyma by three orders of magnitude.
This suggests that the mechanical influence of the membrane on the liver behavior is not negligible.
%
In~\cite{Hollenstein2006}, a local influence of the capsule has been measured using a special aspiration device. The study was then repeated 
in vivo on human patients during the operation, confirming the mechanical importance of the membrane~\cite{Ahn2010,Nava2008}.
%To our best knowledge, no attempt has been made to demonstrate the role of the Glisson's capsule on the global behaviour of 
%the liver, mainly if the organ undergoes large deformations which is often the case during the surgery.

In this paper we present preliminary results of our work on the complete liver model. 
We present a real-time composite model accounting for parenchyma and Glisson's capsule, compatible with previously proposed vascularized model~\cite{Peterlik2012}.
We also show that the capsule has a significant impact on the mechanical response of the organ. 
%The model is based on two different finite element representations for each constituent coupled together.
%The main contribution of the paper is twofold: first, we present a real-time model of the liver, including
%the parenchyma, vascularization and Glisson's capsule. The model is based on three different finite element representations for each constituent,
%linked together via mechanical coupling.
%We show that the model mimics the \emph{local} experiments described in~\cite{Hollenstein2006}.
 
%Second, we use the complete model of the liver to demonstrate the \emph{global} influence of the Glisson's
%capsule via simulation: using a specific model of a porcine liver built from CT contrast-enhanced data, we show that there is a significant 
%difference in the response of the model with and without the capsule in the case when the liver undergoes large deformations. 

%The paper is organized as follows: first we describe the proposed model of liver with capsule. 
%Second, we validate our model in context of local deformations by reproducing the aspiration test described in~\cite{Hollenstein2006}. We conclude by summarizing future steps towards the complete liver model.
%Finally, we demonstrate that in spite of its small thickness, the Glisson's 
%capsule has a global influence on the liver undergoing large deformations. 
