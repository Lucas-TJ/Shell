\section{Methods \& Materials}
\label{sec:methodology}
\vspace{-6pt}

In this section we describe the construction of a composite model based on two components: tetrahedral FE model of the  parenchyma and triangular membrane elements used for the capsule.

\subsection{Parenchyma} %{{{

It is known that the parenchyma exhibits non-linear viscoelastic behaviour \cite{Marchesseau2010}.
However, as we are mainly interested in the static equilibrium, we do not model the time-dependent
phenomena related to viscosity.
%
%However, we employ simpler corotational model as we are not so much
%interested in time-dependent behaviour but rather in static equilibrium
%under certain conditions.
%We also rely on the vascularized model of the parenchyma proposed by
%Peterl\'{i}k et al. \cite{Peterlik2012}.
%
The parenchyma is modeled using corotational finite elements~\cite{Felippa2005}.
Although it relies on linear stress-strain relationship, large displacements including rotations are modeled correctly. 
While in the full non-linear formulation the stiffness matrix relates the forces $\Vec{f}$ and 
displacements $\Vec{u}$ as $\Vec{f} = K(\Vec{u})$, the corotational model 
requires the stiffness matrix $\Mat{K}_0$ of the system to be computed only once before the simulation begins. 
Then, in each step, the motion of each element $e$ is decomposed into rigid rotation $\Mat{R}^e$ and local deformation. 
The rotations are then used to update each local element stiffness matrix as $\Mat{R}^e\Mat{K}_0{\Mat{R}^e}^{\top}$
whereas the deformations are used to compute the linear strain in the local corotational frame.
There are several ways of computing the corotational frame for elements; we rely on
the geometrical method proposed in \cite{Nesme2005}.
% NOTE: This description of corotational method is very simplified and could be extended.

%The model of the vascularization is based on linear beams 
%with local frames of reference~\cite{Duriez2006}; in many aspects it's similar to the
%corotational formulation described above. As such the model also handles geometric
%non-linearities in the deformation. Through a specification of cross section and moments of inertia, 
%the model can account for the specific properties of the blood vessels. 

%The beam-based model of vessels is mechanically coupled to the parenchyma as described in~\cite{Peterlik2012}. 
%The coupling assumes that there is no relative motion between the vessels and the surrounding parenchyma. 
%During the simulation, the nodes of linked beams are first displaced and rotated according to the actual motion of associated tetrahedra. 
%As the deformation of beams results in mechanical response represented by forces and torques, these are propagated back to 
%the tetrahedral FE model. 

%}}}

\subsection{Glisson's Capsule} %{{{
\label{ss:capsuleModel}
The thickness of the Glisson's capsule is relatively small: the values in range of 10-20
$\mu$m have been reported in~\cite{Umale2011}.
It is not possible to model such thin structure with classical tetrahedral
elements, if the real-time aspect of the simulation is to be preserved.
Furthermore, modeling both the tissue and the capsule would require an extremely 
dense mesh to avoid numerical instabilities and thus would significantly
violate the speed requirements imposed for medical simulators.
Instead, modeling the capsule with two-dimensional elements that abstract from the
thickness in the third dimension seems
to be a natural choice. In the elasticity theory, this functionality is usually provided by membrane and shell elements.
Based on the observation of its behaviour, we also
assume negligible bending forces and propose a model based on membrane
elements. 
To maintain simplicity of the composite model we choose simple triangular
elements with constant strain (CST).

The computation of elastic stiffness matrix follows the common derivation
%
\begin{eqnarray}
  \Mat{K}^m & = & \int_V \Mat{B}^T \Mat{E} \Mat{B} dV     \label{mem1} \\
            & = & h \int_A \Mat{B}^T \Mat{E} \Mat{B} dA   \label{mem2} \\
            & = & h A \Mat{B}^T \Mat{E} \Mat{B}           \label{mem3}
\end{eqnarray}
%
where $\Mat{B}$ is the strain-displacement matrix, $\Mat{E}$ the material
matrix, $h$ is the thickness and $A$ area of the element. In the previous
\eqref{mem2} follows from the fact that we assume constant thickness of the
element and \eqref{mem3} follows from the fact that the strain-displacement
matrix is constant in our case. The strain-displacement matrix for the CST
element can be expressed as:
%
\begin{equation}
  \Mat{B} = \frac{1}{2A} \begin{bmatrix}
    y_{23} & 0      & y_{31} & 0      & y_{12} & 0 \\
         0 & x_{32} & 0      & x_{13} & 0      & x_{21} \\
    x_{32} & y_{23} & x_{13} & y_{31} & x_{21} & y_{12}
  \end{bmatrix}
\end{equation}
%
The values $x_{ij} = x_i - x_j$ and $y_{ij} = y_i - y_j$ are computed from
the $x$ or $y$ coordinates of the nodes $i,j$ of the triangular element.
The reader can refer to the respective literature~\cite{Felippa2003} for more thorough
description.

Similarly as with model of parenchyma we use linear elastic material and employ
the corotational formulation for the CST elements.

%}}}

\subsection{Coupling Between Capsule and\\Parenchyma} %{{{
The literature reports high cohesion between capsule and parenchyma.
Based on this property we assume there is no relative motion of the capsule \wrt\ the parenchyma.
Although an arbitrary surface mesh could be used to model the capsule, we exploit 
the fact that the parenchyma is modeled by tetrahedral elements having
triangular faces. Thus, as the boundary of the volumetric mesh is already
triangulated, we simply employ the triangles on the mesh surface to model the capsule.

Using directly the boundary of the tetrahedral mesh does not only solve the
problem of building the surface mesh, but has one more advantage: the nodes
of the triangular mesh coincide with the nodes of the tetrahedral mesh, so no projection of one mesh onto the other is needed.
Moreover, the stiffness matrices for capsule and parenchyma are then easily assembled together.
% and solved as one system.
%
%\CD{The following may appear as "trivial"... maybe we can remove this part at the end if we need space} 
Without the loss of generality we can assume the tetrahedron consists of
nodes $p_1, p_2, p_3$ and $p_4$ and the boundary triangle has nodes $p_1, p_2$
and $p_3$. We can reorder the degrees of freedom so that the stiffness
matrix $\Mat{K}^t$ for the tetrahedron can be written as:
%
\begin{equation}
  \Mat{K}^t = \left[\begin{array}{c|c}
      \Mat{K}^t_{1-3,1-3} & \Mat{K}^t_{1-3,4} \\
      \hline
      \Mat{K}^t_{4,1-3} & \Mat{K}^t_{4,4} \\.
  \end{array}\right]
\end{equation}
%
Hence, the assembled stiffness matrix for the element is
%
\begin{equation}
  \Mat{K} = \left[\begin{array}{c|c}
      \Mat{K}^t_{1-3,1-3} & \Mat{K}^t_{1-3,4} \\
      \hline
      \Mat{K}^t_{4,1-3} & \Mat{K}^t_{4,4} \\
  \end{array}\right]
  +
  \left[\begin{array}{c|c}
      \Mat{K}^m & 0 \\
      \hline
      0 & 0 \\
  \end{array}\right]
\end{equation}
%
where $\Mat{K}^m$ is the stiffness matrix of the triangular membrane.

The resulting system of linear equations is solved by direct solver based on Cholesky decomposition.

%Alternatively, one can use a mechanical coupling similar to the one used in
%\cite{Peterlik2012} to be able to use an arbitrary surface mesh. Nevertheless,
%for conforming triangular mesh both methods lead to the same solution.

%}}}

%}}}


\section{Methods} %{{{
The model presented in the previous section was implemented in 
SOFA\footnote{www.sofa-framework.org} and a set of
numerical simulations was performed.
In this section we provide comparison of local deformations with the
results reported in literature to validate the method.
%Second, to show that in spite of its very small thickness the membrane cannot be
%neglected even in the context of global deformations and its overall
%stiffness plays an important role, the model of
%the complete liver was subjected to global deformations.

During the contact with an instrument such as probe, needle, scalpel and others,
specific deformations take place in the vicinity of
the instrument. This type of deformation may not necessarily induce the
deformation of the object as a whole and therefore can be considered as
local. Correct material properties are not only important to quantify the
displacement, but also play an important role in capturing the correct area of
the deformation or its profile near the instrument.

A good example of such a local deformation is the aspiration test
where the response of liver exposed locally to a negative pressure is measured.
The aspiration device consists of a tube having 1\,cm in diameter and allows to
control the pressure inside the tube. The test is performed by
attaching the tube to the tissue and measuring the tissue response. We
set up a simulation in SOFA to reproduce the experiment (see
Fig.~\ref{fig-aspiration1}): we meshed a 15$\times$15$\times$15\,mm$^3$ 
cube representing the tissue resulting in 2648 tetrahedra. Then we attached a 1\,cm tube 
and applied a pressure of 3\,kPa inside the tube. 

Since the tube is in direct contact with the tissue, uni-lateral constraints with friction were chosen 
to model this interaction properly. We opted for a method based on \emph{non-linear complementarity problem}  (NLCP)
where the non-linearity is introduced due to the friction. The NLCP
allows for solution of the Signorini's problem to avoid any interpenetration between the colliding 
objects (see~\cite{Duriez2006b} for details). Since NLCP requires explicitly the computation of compliance matrix which 
is homogeneous to the inverse of the stiffness matrix, we employed a direct solver based on LDL decomposition to 
solve the system and compute the inverse matrices. 