\begin{frontmatter}%
\begin{abstract}%
Accurate biomechanical modeling of liver is of a paramount interest for pre-operative planning or computer-aided per-operative guidance. 
However, such simulations remain a challenging task, since the organ is composed of three constituents: parenchyma, vascularization and Glisson's capsule, 
each having different mechanical properties. 
%
%While a real-time simulation of vascularized liver has been already described in~\cite{Peterlik2012}, the Glisson's capsule has not been included in the modeling. 
%In~\cite{Hollenstein2006,Ahn2010} it is shown that the capsule plays an important role mainly in the vicinity of the surgical tools. 
%Therefore, the accuracy of the liver response in the vicinity of the surgical tool requires correct modeling of the capsule as a component of the liver model. 
%
In this paper we propose a complete liver model, where the parenchyma is modeled as a finite element volume, the vessels are modeled as series of beam elements, and the   corotational membrane elements based on constant-strain formulation, which are coupled mechanically to the underlying tetrahedra. 
We show that using this model we are able to reproduce accurately biomechanical tests, while maintaining the real-time aspect of the simulation. 

%The measures performed on the capsule report stiffness which is significantly higher than that of the parenchyma, however, the thickness of the membrane does not exceed 100$\mu$m. 
%For this reason, it is not possible to model the capsule with standard volume elements usually employed for the parenchyma. 
%In our approach, we rely of corotational membrane elements based on constant-strain formulation, which are coupled mechanically to the underlying tetrahedra. 

\end{abstract}%
% 
\begin{keyword}%
Simulation , modeling, liver, biomechanics
\end{keyword}%
% 
\end{frontmatter}%