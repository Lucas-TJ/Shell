% http://www.miccai2013.org/submission_guideline.html
\documentclass{llncs}

\usepackage{makeidx}  % allows for indexgeneration

\begin{document}
%
%\frontmatter          % for the preliminaries
%
\mainmatter              % start of the contributions
%
\title{TODO: Title} % TODO
%
\titlerunning{TODO: Title}  % abbreviated title (for running head)
%                                     also used for the TOC unless
%                                     \toctitle is used
%
\author{Anonymous}
%\author{%
%Tom\'a\v{s} Golembiovsk\'y\inst{1,2} \and%
%Igor Peterl\'ik\inst{3} \and%
%%Christian Duriez\inst{2} \and%
%Stephan\'e Cotin\inst{2,3}%
%}
%
%\authorrunning{Tom\'a\vs Golembiovsk\'y et al.} % abbreviated author list (for running head)
%
%\institute{%
%Faculty of Informatics, Masaryk University, Brno, Czech Republic\and%
%INRIA Lille -- Nord, France\and%
%IHU, Strasbourg, France%
%}
%
\maketitle

% TODO
\begin{abstract}
TODO: Abstract
\keywords{liver, Glisson's capsule, deformation modeling, membrane}
\end{abstract}

%\section{Introduction}
\section{...}

TODO

The Glisson's capsule \cite{Umale2013} has in order of three higher
stiffness than the parenchyma. As such it plays an important role in the
deformation of the liver \cite{Ahn2010,Hollenstein2006}. Evaluation of the
liver as a homogeneous material can lead to overestimation of the
mechanical properties up to the factor of 3 \cite{Hollenstein2006}.

% -- TODO: properties of the capsule

We propose to further extend the tetrahedral based model of the parenchyma
with vascularized model \cite{Peterlik2012} by the layer of membrane elements.
(Modelling the capsule with tetrahedral elements is unfeasible -- many
elements or largely deformed elements.)

Based on the small thickness it seems reasonable to assume that no bending
strains limmit the deformation of the capsule. Therefore modelling the
capsule with membrane element only should be adequate.

% -- TODO: what is membrane element
% -- 



\section{Experiments/Evaluation}

\section{Results}

\section{Conclusion}



%
% ---- Bibliography ----
%

\bibliographystyle{splncs03}
\bibliography{bibdata}


\end{document}
% vim:set et sw=2 tw=75 fdm=marker fdl=1 fdc=4 isk+=_,-:
