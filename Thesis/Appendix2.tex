\chapter{The weighted residual method}
\label{appendix2}

The \emph{method of weighted residuals} is an approximation technique for solving differential equations. A subclass of this method, the \emph{Galerkin method of weighted residuals}, is often used to derive the element equations for the finite element method. 

Let us suppose that we have a linear differential operator $ D $ acting on a function $ u $ to produce a function $ p $:
\begin{equation}
D(u(x)) = p(x).
\end{equation}
We wish to approximate $ u $ by a function $ \tilde{u} $, which is a linear combination of basis functions chosen from a linearly independent set, That is: 
\begin{equation}
u \approx \tilde{u} = \sum_{i=1}^n a_i \phi_i .
\end{equation}
Now, when substituted into the differential operator $ D $, the result of the operation is not, in general, $ p(x) $. Hence, an error or \emph{residual} will exist:
\begin{equation}
R(x) = D(\tilde{u}(x)-p(x)) \neq 0.
\end{equation}
The idea behind the weighted residual method is to force the residual to zero in some average sense over the domain:
\begin{equation}
\int_X R(x) W_i(x) dx = 0, \quad i = 1, 2, \ldots, n
\end{equation}
where the number of \emph{weight functions} $ W_i $ is exactly equal to the number of unknown constants $ a_i $ in $ \tilde{u} $. The result is a set of $ n $ algebraic equations for the unknown constants $ a_i $. Depending on the choice for $ W_i $, several sub-methods can be derived: collocation, sub-domain, least squares, Ritz, Galerkin etc. For instance, the weight function used by the Galerkin method \TODO{need to check this: Wi are the shape functions?} is:
\begin{equation}
W_i = \pd{\tilde{u}}{a_i}.
\end{equation}